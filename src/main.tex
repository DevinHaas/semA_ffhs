%! Author = devinhasler
%! Date = 26.11.2024
\documentclass{ffhsthesis}
\usepackage[utf8]{inputenc}
\usepackage{biblatex}
\usepackage{FFHS}
\usepackage{hyperref}
\usepackage{booktabs}
\usepackage{multirow}
\usepackage{enumitem}
\usepackage{parskip}
\usepackage{scrlayer-scrpage}
\usepackage[autostyle]{csquotes}

% Set header height to avoid fancyhdr warning
\setlength{\headheight}{46.75916pt}


\ihead{\includegraphics[height=0.7cm]{images/FFHS_Logo}} % Left header (logo)
\ohead{Devin Hasler} % Right header (name)


\addbibresource{literature.bib} % Hinzufügen der Referenzen

\begin{document}
    \dokumentTyp{Semesterarbeit}
    \studiengang{Informatik}
    \title{Entwicklung eines nativen Moduls zur Screen-Time-Überwachung in einer React-Native-Anwendung: Eine Intervention zur Reduzierung problematischer Smartphone-Nutzung}
    \subtitle{(evt. Untertitel, Foto usw.)} % optional
    \author{Devin Hasler}
    % \date{}
    \wohnort{3423 Lützelflüh, Datum}
    %\referent{Name des Referenten\\ Titel\\Unterrichtetes Fach}
    \referent{Vorname Name\\ Titel\\Dozent für Fach}
    \eingereichtBei{Tobias Häberlein\\Departement Informatik\\Departementsleiter}

    \maketitle


    \begin{zusammenfassung}
        Hier kommt die Zusammenfassung
    \end{zusammenfassung}
    \begin{abstract}
        Hier kommt das Abstract
    \end{abstract}
    \tableofcontents

    \begin{abkuerzungen}[MUSTER] % Das Muster dient zur Bestimmung der Einrueckungstiefe
        \item[Hrsg.] Herausgeber
        \item[NN] nomen nominandum (Namen nicht bekannt)
        \item [o.J.] ohne Jahrgang
    \end{abkuerzungen}


    \startThesis % Befehl muss vor dem ersten chapter stehen (Seitennummerierung!)




    \chapter{Vorwort}\label{ch:vorwort}

Hier kommt das Vorwort
    \chapter{Einleitung}\label{ch:einleitung}

Das ist ein Test, ob hier alles funktioniert
    \chapter{Material}\label{ch:material}


    \include{chapters/methodik}
    \chapter{Auswertung und Resultate}\label{ch:auswertung und resultate}


    \include{chapters/diskussion}
    \include{chapters/fazit}
    \chapter{Ausblick}\label{ch:ausblick}



    % ==============================
    % Anhang
    \clearpage\appendix
    \renewcommand{\thesection}{A}
    %! suppress = MissingImport
\section{Literatur und Quellenverzeichnis}
\label{sec:references}

\printbibliography[heading=subbibliography,title={~}] % Do not print any additional title

    \listoffigures
    \listoftables
    \lstlistoflistings
    \chapter*{Hilfsmittelverzeichnis}
    \addcontentsline{toc}{chapter}{Hilfsmittelverzeichnis}

    \begin{table}[h!]
        \centering
        \begin{tabular}{|p{5cm}|p{5cm}|p{4cm}|} \hline
        \textbf{Welches Hilfsmittel wurde eingesetzt?} & \textbf{Wozu wurde das Hilfsmittel eingesetzt?}  & \textbf{Betroffene Stellen}\\ \hline
        Bezahltes Lektorat & Rechtschreibkorrektur & Gesamtes Dokument \\ \hline
        Google Translate & Übersetzung von Textpassagen & Kapitel 5.3, Seite 25-26 \\ \hline
        ChatGPT & Kapitelstruktur & Kapitel 5, Seiten 23-39 \\ \hline
        &  & \\ \hline
        \end{tabular}
        \label{tab:hilfsmittelverzeichnis}
    \end{table}
    \include{chapters/anhang}



    \chapter*{Selbstständigkeitserklärung}
    Ich erkläre hiermit, dass ich diese Thesis selbständig verfasst
    und keine andern als die angegebenen Quellen benutzt habe.
    Alle Stellen, die wörtlich oder sinngemäss aus Quellen entnommen wurden,
    habe ich als solche kenntlich gemacht.
    Ich versichere zudem, dass ich bisher
    noch keine wissenschaftliche Arbeit mit gleichem oder ähnlichem Inhalt an der
    Fernfachhochschule Schweiz oder an einer anderen Hochschule eingereicht habe.
    Mir ist bekannt, dass andernfalls die Fernfachhochschule Schweiz zum Entzug
    des aufgrund dieser Thesis verliehenen Titels berechtigt ist.
    \vspace{4cm}
    \noindent
    \signature{3cm}{2cm}{signatureShort}
    \hrule \ \\[-0.5ex]
    Lützelflüh, 23 June 2024, Devin Hasler

\end{document}
